\documentclass[12pt]{article} 

\usepackage{geometry}
\geometry{a4paper} 

\usepackage{graphicx} 
\usepackage{enumitem}
\usepackage{booktabs}

\usepackage{float} 
\usepackage{wrapfig} 

\usepackage{amsmath}
\usepackage{amsfonts}
\usepackage{amssymb}
\usepackage{dsfont}

\usepackage{xcolor}
\usepackage{listings}
\usepackage{caption}
\DeclareCaptionFont{white}{\color{white}}
\DeclareCaptionFormat{listing}{%
  \parbox{\textwidth}{\colorbox{gray}{\parbox{\textwidth}{#1#2#3}}\vskip-2pt}}
\captionsetup[lstlisting]{format=listing,labelfont=white,textfont=white}
\lstset{frame=lrb,xleftmargin=\fboxsep,xrightmargin=-\fboxsep}

\linespread{1.2} 
\setlength{\parskip}{\baselineskip} % vertical spaces
\setlength\parindent{0pt} % remove all indentation from paragraphs


\usepackage{ntheorem}
\usepackage{mdframed}

\theoremstyle{nonumberbreak}
\theoremheaderfont{\bfseries}
\newmdtheoremenv[%
linecolor=gray,leftmargin=10,%
rightmargin=10,
backgroundcolor=gray!20,%
innertopmargin=0pt,%
ntheorem]{theorem}{}




\begin{document}

\title{\textbf{Limit Theorems for \\ Renewal Processes}}
\author{Hyunwoo Gu}
\date{}

\maketitle

%----------------------------------------------------------------------------------------
%	Supplementary
%----------------------------------------------------------------------------------------
\section{The discrete-time Renewal Equation}

\subsection{The Renewal Theorem}

Recall that 

$$
M(t) = \mathbb{E} [ N(t) ] = \sum_{j=1}^\infty F_j(t)
$$

where $F_j(t) = Pr[ S_j \le t]$ for $t \ge 0$. We call $M(t)$ a \textbf{renewal function}.


Let us now show that the renewal function $M(t)$ satisfies the equation

$$
\begin{aligned}
M(t) &= F(t) + \int_0^t M(t-y) dF(y), \ \ t \ge 0 \\[8pt]
\Leftrightarrow M(t) &= F(t) + F \ast M(t), \ \ t \ge 0
\end{aligned}
$$



\textbf{Renewal Equations}

An integral equation of the following form is called a \textbf{renewal equation}. 

$$
A(t) = a(t) + \int_0^t A(t-x) dF(x), \ \ t \ge 0
$$


\begin{theorem}
\textbf{Theorem 4.1}. Suppose $a$ is a bounded function. Then there uniquely exists $A$ bounded on finite intervals satisfying

$$
A(t) = a(t) + \int_0^t A(t-x) dF(y)
$$


Namely, 

$$
A(t) = a(t) + \int_0^t a(t-x) dM(x)
$$

where $M(t) := \sum_{k=1}^\infty F_k(t)$ : the renewal function. 
\end{theorem}


\textbf{Proof}. We verify first that $A$ specified fulfilles the requisite boundedness properties and solves 



$$
\begin{aligned}
A &= a + F \ast a + F_2 \ast (a + F \ast A) \\[8pt]
&= a + F \ast a = F_2 \ast a + F_3 \ast A \\[8pt]
&= a + \left( \sum_{k=1}^{n-1} F_k \right) \ast a + F_n \ast A
\end{aligned}
$$


Next observe that 


\begin{theorem}
\textbf{Theorem 4.2}. Let $\{ X_t \}$ be a renewal process with $\mu = \mathbb{E} X_1 < \infty$. Then

$$
\mathrm{lim}_{t \to \infty} \frac{1}{t} M(t) = \frac{1}{\mu}
$$
\end{theorem}


\textbf{Proof}. Note that $t < S_{N(t) + 1}$. Thus 

$$
t < \mathbb{E} \left[ S_{N(t) + 1} \right] = \mu \left[ 1 + M(t) \right]
$$

and therefore

$$
\frac{1}{t} M(t) > \frac{1}{\mu} - \frac{1}{t}
$$

It follows that

$$
\mathrm{lim inf}_{t \to \infty} \frac{1}{t} M(t) \ge \frac{1}{\mu}
$$

To establish the opposite inequality, let $c > 0$ be arbitrary, and set

$$
X_i^c = \begin{cases}
X_i & X_i \le c \\
c   & X_i > c
\end{cases}
$$

and consider the renwal process having lifetimes $\{ X_i^c \}$. Let $\{ S_n^c \}$ and $\{ N^c(t) \}$ denote the waiting times and counting process, respectively, for this \textbf{truncated renewal process} generated by $\{ X_i^c \}$. Since the random variables $X_i^c$ are uniformly bounded by $c$, it is clear that $t + c \ge S^c_{N^c(t) + 1}$, and therefore

$$
t + c \ge \mathbb{E} \left[ S^c_{N^c(t) + 1} \right] = \mu^c \left[ 1 + M^c (t) \right]
$$


where 


From $N^c(t) \ge N(t)$ from $X_i^c \le X_i$, and $M^c (t) \ge M(t)$, we have

$$
\begin{aligned}
t + c &\ge \mu^c \left[ 1 + M(t) \right] \\[8pt]
\therefore \ \ \frac{1}{t} M(t) &\le \frac{1}{\mu_c} + \frac{1}{t} \left( \frac{c}{\mu^c} -1 \right)
\end{aligned}
$$


Hence 

$$
\mathrm{lim sup}_{t \to \infty} \frac{1}{t} M(t) \le \frac{1}{\mu^c}, \ \ \forall c > 0
$$

since 

$$
\begin{aligned}
\mathrm{lim}_{c\to \infty} \mu^c &=\mathrm{lim}_{c \to \infty} \int_0^c \left[ 1- F(x) \right] dx  \\[8pt]
&= \int_0^\infty [1-F(x)] dx = \mu
\end{aligned}
$$

while the left-hand side is fixed, we deduce that

$$
\mathrm{lim sup}_{t \to \infty} \frac{1}{t} M(t) \le \mathrm{lim}_{c \to \infty} \frac{1}{\mu^c} = \frac{1}{\mu}
$$



\section{The Continuous-time Renewal Theorem}

$$
M(t + h) - M(t) \to \frac{h}{\mu}, \ \ t \to \infty
$$

In words, the expected number of renewals in an interval with length $h$ i sapproximately $h/\mu$, provided the process has been in operation for a long duration. 



\begin{theorem}
\textbf{Definition 5.1}. A point $\alpha$ of a distribution function $F$ is called a \textbf{point of increase} if for a positive $\epsilon$  

$$
F(\alpha + \epsilon) - F(\alpha - \epsilon) > 0
$$
\end{theorem}

A distribution function is \textbf{arithmetic} if there exists $\lambda >0$ such that $F$ exhibits points of increase exclusively among $0, \pm \lambda, \pm 2\lambda, \cdots$. The largest such $\lambda$ is called the \textbf{span} of $F$.

Note that $F$ that has a continuous part is \textbf{not arithmetic}. The distribution function of a discrete RVs having posiible values $0,1,2,\cdots$ is arithmetic with span $1$. 




\begin{theorem}
\textbf{Definition 5.2}. A point $\alpha$ of a distribution function $F$ is called a \textbf{point of increase} if for a positive $\epsilon$  

$$
F(\alpha + \epsilon) - F(\alpha - \epsilon) > 0
$$
\end{theorem}

\textbf{Every monotonic function} $g$ \textbf{which is absolutely integrable in the sense that}

$$
\int_0^\infty |g(t)| dt < \infty
$$ 

is \textbf{directly Riemann integrable}. Manifestly, all finite linear combinations of monotone functions satisfying the above are also directly Riemann integrable. 


\begin{theorem}
\textbf{Theorem 5.1. (The Basic Renewal Theorem)}. Let $F$ be the distribution function of a positive random variable with mean $\mu$. Suppose that $a$ is directly Riemann integrable and that $A$ is the solution of the renewal equation

$$
A(t) = a(t) + \int_0^t A(t-x) dF(x)
$$

(i) If $F$ is not arithmetic, then 

$$
\mathrm{lim}_{t\to\infty} A(t) = \begin{cases}
\frac{1}{\mu} \int_0^\infty a(x) dx & \mu < \infty \\
0 & \mu = \infty
\end{cases}
$$

(ii) If $F$ is arithmetic with span $\lambda$, then $\forall c>0$, 

$$
\mathrm{lim}_{t\to\infty} A(t) = \begin{cases}
\frac{\lambda}{\mu} \sum_{n=0}^\infty a(c + n\lambda) & \mu < \infty \\
0 & \mu = \infty
\end{cases}
$$
\end{theorem}

In a simpler form, we say that







Combining 



An obvious generalization of integer-valued inter-renewal intervals is that of inter-renewals that occur only at integer multiples of some real number $\lambda >0$. Such a distribution is called an \textbf{arithmetic distribution}. The \textbf{span} of an arithmetic distribution is the largest number $\lambda$ s.t. this property holds. 






\section{Applications of the Renewal Theorem}

\subsection{Limiting Distribution of the Excess Life}



\subsection{Asymptotic Expansion of the Renewal Function}

Suppose $F$ is a nonarithmetic distribution with a finite variance $\sigma^2$. Under these assumptions we will determine the second term in the asymptotic expansion of $M(t)$ by proving

$$
\mathrm{lim}_{t\to\infty} \{ M(t) - \mu^{-1}t \} = \frac{\sigma^2 - \mu^2}{2\mu^2}
$$



\section{Applications of the Renewal Theorem}

\subsection{Delayed Renewal Processes}

$$
\begin{aligned}
M_D(t) &= \int_0^\infty \mathbb{E}[N(t) | X_t = x ] dG(x) \\[8pt]
&= \int_0^t \{ 1 + M(t-x) \} dG(x) \\[8pt]
&= G(t) + \int_0^t M(t-x) dG(x) \\[8pt]
&= G(t) + \int_0^t G(t-x) dM(x) \\[8pt]
\end{aligned}
$$



\subsection{Stationary Renewal Processes}

A delayed renewal process for which the first life has the distribution
function

$$
G(x) = \mu^{-1} \int_0^x \{ 1 - F(y) \} dy
$$




\subsection{Alternating and Markov Processes}

An \textbf{alternating renewal process} is a sequence $Y_1, Y_2, \cdots$ of independent RVs, where

$$
\begin{aligned}
Y_1, Y_{r+1}, Y_{2r+1}, \cdots &\sim F_1 \\[8pt]
Y_2, Y_{r+2}, Y_{2r+2}, \cdots &\sim F_2 \\[8pt]
\cdots &\vdots \cdots \\[8pt]
Y_r, Y_{2r},  Y_{3r},   \cdots &\sim F_r \\[8pt]
\end{aligned}
$$



\subsection{Central Limit Theorem for Renewals}




\subsection{Characterization of the Poisson Process}

Poisson process is a special type of a renewal process. Let $\{ X_k \}$ be a renewal process with $\mathbb{E} [X_k] := \mu < \infty$ and $F(x) = P \{ X_k \le x \}$. Assuming $F(0) = 0$, define

$$
F_t(x) = \begin{cases}
F(x) & 0 \le x < t \\[8pt]
1 & t \le x
\end{cases}
$$

which is basically the distribution function for $min \{ X_k, t \}$



\begin{theorem}
\textbf{Theorem 8.1.}.
(a) If there exists a sequence $\{ t_j \}$, where $t_j \to \infty$ as $j \to \infty$, and for which the current life $\delta_t$ satisfies 

$$
F_{t_j} (x) = Pr[ \delta_{t_j} \le x ], \forall x
$$

then $F$ is an \textbf{exponential distribution}.

(b) If there exists a sequence $\{ t_j \}$,  where $t_j \to \infty$ as $j \to \infty$, and for which

$$
F(x) = Pr [ \gamma_{t_j} \le x], \forall x
$$

then $F$ is an \textbf{exponential distribution}.
\end{theorem}


\textbf{Proof}. By the result of (6.5), i.e. 


the limiting distribution of the current life $\delta_t$ i s

$$
\mathrm{lim}_{t\to\infty} Pr\{ \delta_t > y \} = \mu^{-1} \int_y^\infty [1 - F(z)] dz
$$


Letting $t$ increase along $t_j$ with due account of the hypothesis of the theorem, we derive the functional equation

$$
1 - F(y) = \mu^{-1} \int_y^\infty [1- F(z)] dz
$$

The right-hand side is clearly differentiable in $y$, yielding the elementary first-order DE

$$
\frac{d}{dy} [1 - F(y)] = - \frac{1}{\mu} [1-F(y)]
$$

whose solution, subject to $F(0) = 0$, is

$$
1 - F(y) = e^{-\lambda y}, \lambda := 1/\mu
$$







\subsection{Superposition of Renewal Processes}

\begin{theorem}
\textbf{Definition 9.1}. The triangular array $\{ N_{ni}(t) \}$ is called \textbf{infinitesimal} if for every $t \ge 0$,

$$
\mathrm{lim}_{n \to \infty} \mathrm{max}_{1 \le i \le k_n} F_{ni} (t) = 0
$$
\end{theorem}



\begin{theorem}
\textbf{Theorem 9.1}. Let $\{ N_{ni}(t) \}$ be an infinitesimal array of renewal processes with superposition $N_n(t)$. Then

$$
\mathrm{lim}_{n\to\infty} Pr \{ N_n(t) = j \} = \frac{e^{-\lambda t (\lambda t)^j}}{j!}, \ \ j=0,1,2,\cdots
$$

if and only if

$$
\mathrm{lim}_{n\to\infty} \sum_{i=1}^{k_n} F_{ni}(t) = \lambda t
$$

\end{theorem}

\textbf{Proof}. 

(1) \textit{Necessity}. For $j=0$, we obtain 

$$
\mathrm{lim}_{n\to \infty} Pr \{ N_t(t) = 0 \} = e^{-\lambda t}
$$

or, equivalently,

$$
\
$$


(1) \textit{Sufficiency}. Let us follow an induction on $m$ to show

$$
\mathrm{lim}_{n\to\infty} Pr \{ N_n(t) = m \} = \frac{e^{-\lambda t (\lambda t)^m}}{m!}, \ \ m=0,1,2,\cdots
$$

For $m=0$, 





\textbf{Example 1}. Suppose $F(t)$ is a distribution function for which $F(0) = 0$, $F'(0) = \lambda >0$. Let

$$
F_{ni}(t) = F(t/n), \ \ i=1,\cdots,n
$$

and, for all $n$, let $N_{ni} (t), i=1,\cdots,n$ be independent renewal counting processes with interoccurence distribution $F_{ni}$. Then $N_{ni} (t)$ is a triangular array. Furthermore, since 




Hence, the distribution of the superposition $N_n(t)$ converges to the Poisson process. 





\begin{theorem}
\textbf{Theorem 9.2}. Let $N_1(t)$ and $N_2(t)$ be two independent renewal processes with the same interoccurrence distribution $F$ having mean $\mu$. Let $N(t) = N_1(t) + N_2(t)$. If $N(t)$ is also a renewal process, then $N_1(t), N_2(t), N(t)$ are all Poisson.
\end{theorem}


\textbf{Proof}. Let $H$ be the interoccurrence distribution for $N(t)$. Then 

$$
\begin{aligned}
1 - H(x) &= Pr[N(x) = 0] \\[8pt]
&= Pr[N_1(x)=0, N_2(x)=0] \\[8pt]
&= [1-F(x)]^2
\end{aligned}
$$

Let 



$$
\frac{1}{v} \int_x^\infty [1-H(y)]dy = \frac{1}{\mu^2} \left\{ \int_x^\infty [1-F(y)] dy \right\}^2
$$

where $v := \int_0^\infty [1-H(y)] dy$. Both sides are differentiable with respect to $x$, and earlier we noted $1-H(x) = [1-F(x)]^2$.  




\end{document}




