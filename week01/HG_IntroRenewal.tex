
\documentclass[12pt]{article} 

\usepackage{geometry}
\geometry{a4paper} 

\usepackage{graphicx} 
\usepackage{enumitem}
\usepackage{booktabs}

\usepackage{float} 
\usepackage{wrapfig} 

\usepackage{amsmath}
\usepackage{amsfonts}
\usepackage{amssymb}
\usepackage{dsfont}

\usepackage{xcolor}
\usepackage{listings}
\usepackage{caption}
\DeclareCaptionFont{white}{\color{white}}
\DeclareCaptionFormat{listing}{%
  \parbox{\textwidth}{\colorbox{gray}{\parbox{\textwidth}{#1#2#3}}\vskip-2pt}}
\captionsetup[lstlisting]{format=listing,labelfont=white,textfont=white}
\lstset{frame=lrb,xleftmargin=\fboxsep,xrightmargin=-\fboxsep}

\linespread{1.2} 
\setlength{\parskip}{\baselineskip} % vertical spaces
\setlength\parindent{0pt} % remove all indentation from paragraphs


\usepackage{ntheorem}
\usepackage{mdframed}

\theoremstyle{nonumberbreak}
\theoremheaderfont{\bfseries}
\newmdtheoremenv[%
linecolor=gray,leftmargin=10,%
rightmargin=10,
backgroundcolor=gray!20,%
innertopmargin=0pt,%
ntheorem]{theorem}{}




\begin{document}

\title{\textbf{Intro \& Renewal Process}}
\author{Hyunwoo Gu}
\date{}

\maketitle

%----------------------------------------------------------------------------------------
%   Section 1
%----------------------------------------------------------------------------------------
\section{Introduction to Stochastic Processes}




\subsection{Introduction}

\textbf{$\sigma$-algebra} on a set $\Omega$ is a collection $F$ of subsets of $\Omega$ :

\begin{itemize}
	\item \textbf{Includes $\Omega$ itself},
	\item Closed under \textbf{complement}
	\item Closed under \textbf{countable unions}
\end{itemize}

where the pair $(\Omega, F)$ is called a \textbf{measurable space} or \textbf{Borel space}. More rigorously, letting $X$ be some set, and $\mathcal{P}(X)$ be its power set. Then $\Sigma \subseteq \mathcal{P}(X)$ is called \textbf{$\sigma$-algebra} on $X$ if

\begin{itemize}
	\item $X \in \Sigma$
	\item If $A \in \Sigma$, then $X - A \in \Sigma$ 
	\item If $A_1, A_2, \cdots \in \Sigma$, then $A_1 \cup A_2 \cup \cdots \in \Sigma$
\end{itemize}

where from De Morgan's laws $\Sigma$ is also closed under countable intersections. For example, $\{ \Omega, \emptyset \}$ is the smallest possible $\sigma$-algebra on $\Omega$, whereas the largest possible $\sigma$-algebra on $\Omega$ is $2^\Omega := \mathcal{P}(\Omega)$. Elements of the $\sigma$-algebra, i.e. an ordered pair $(\Omega, F)$ is called a \textbf{measurable set}.


For example, let $(\Omega, \mathcal{F}, P)$ be a probability space in Bernoulli scheme, where $(a_1, \cdots, a_n), a_i \in \{0,1\}$.

\begin{itemize}
	\item $\Omega$, \textbf{sample space}: $\{0,1 \}^n$, $|\Omega| = 2^n$
	\item $\mathcal{F}$, \textbf{filtration}: $|\mathcal{F}| = 2^{2^n}$, since it is the power set.
	\item $P$, the \textbf{probability} measure
\end{itemize}



\subsection{Stochastic functions}


Let $(\Omega, \mathcal{F}, \mathbb{P})$ be a \textbf{probability space}(or probability triple). \textbf{Random variable} is a function $\xi : \Omega \to \mathbb{R} $ such that $\xi^{-1} (B) \in \mathcal{F}, \forall B \in \mathcal{B}(\mathbb{R})$


For time $T$, $X : T \times \Omega \to \mathbb{R}$ is \textbf{random function} if $X(t, \cdot)$ is a random variable on $(\Omega, \mathcal{F}, \mathbb{P})$ for all $t$.


$T = \mathbb{R}_{+}$ case is called \textbf{random process}, whereas $T = \mathbb{R}_{+}^n$ case is called \textbf{random field}, where

\begin{itemize}
	\item Discrete time random process, $T = \mathbb{N}$ or $\mathbb{Z} $
	\item Continuous time random process, $T = \mathbb{R}_{+}$ or $\mathbb{R} $
\end{itemize}

Note that \textbf{any stochastic process at any fixed time} is \textbf{a random variable}. 

Let $X : T \times \Omega \to \mathbb{R}$, $T = \mathbb{R}_{+}$. Let a \textbf{finite-dimensional distribution} $(X_{t_1}, X_{t_2}, \cdots, X_{t_n})$ for $t_1, \cdots, t_n \in \mathbb{R}$ be given. A \textbf{trajectory}(path) is $T \to \mathbb{R}$ for a fixed $\omega$, $(X_{t_1}(\omega), X_{t_2}(\omega), \cdots, X_{t_n}(\omega))$.

For example, $X_t = \xi t$, for $P(\xi = 1) = P(\xi = 2) = 1/2$ has only two possible trajectories, since the only source of randomness if $\xi$. Note that for $t_1, t_2$, 

$$
\begin{aligned}
P(X_{t_1} \le x_1, X_{t_2} \le x_2) &=
\begin{cases}
0 & min(x_1/t_1, x_2/t_2) < 1 \\
1/2 & min(x_1/t_1, x_2/t_2) \in [1,2) \\
1 & min(x_1/t_1, x_2/t_2) \in [2,\infty) \\
\end{cases} \\[8pt]
\end{aligned}
$$


\subsection{Renewal processes}

Let $S_0 = 0$, $S_n = S_{n-1} + \xi_n$, where $\xi_i >0$ IID. Letting $N_t := \mathrm{argmax}_k \{S_k \le t\}$, 

$$
F \to \mathbb{E} N_t
$$

For $X \perp Y$, we have the \textbf{convolution}

$$
F_{X+Y}(x) = F_X \ast F_Y := \int_{\mathbb{R}} F_X(x-y) dF(y)
$$

\begin{theorem}
\textbf{Theorem}. For $S_n = S_{n-1} + \xi_n$, $\xi_i$ IID,
\begin{itemize}
	\item $u(t) = \sum_{n=1}^\infty F^{n\ast} (t) < \infty$
	\item $u(t) = \mathbb{E} N_t$
\end{itemize}
\end{theorem}

\subsection{Laplace transform}



For $f : \mathbb{R}_+ \to \mathbb{R}$, \textbf{Laplace transform} is defined as 

$$
\mathcal{L}_f(s) = \int_0^\infty e^{-sx} f(x) dx
$$

\begin{itemize}
	\item $f$ : density of $\xi$, then $\mathcal{L}_f(s) = m(s) = \mathbb{E}[e^{-s\xi}]$
	\item $\mathcal{L}_{f_1 \ast f_2} (s) = \mathcal{L}_{f_1} (s) \mathcal{L}_{f_2} (s)$ 
	\item $F$ : $\mathcal{L}_F(s) = \mathcal{L}_p(s)/s$
\end{itemize}

For the last property,

$$
LHS = -\int_{\mathbb{R}_+} F(x) \frac{d(e^{-sx})}{s} = - \frac{F(x)e^{-sx}}{s} \Vert_0^\infty + \frac{1}{s} \int_{\mathbb{R}_+}p(x) e^{-sx}dx = \frac{1}{s} \int_{\mathbb{R}_+}p(x) e^{-sx}dx
$$


Consider

$$
F \to \mathbb{E} N_t^{-}
$$

where 

$$
\begin{aligned}
\mathbb{E} N_t &= u(t) = \sum_{n=1}^\infty F^{n\ast} (t) \\[8pt]
&= F(t) + \sum_{n=1}^\infty F^{n\ast} (t) \ast F(t) \\[10pt]
u&= F + u \ast F = F + u \ast p
\end{aligned}
$$

where $\int_\mathbb{R} u(x-y) dF(y) = \int_\mathbb{R} u(x-y) p(y) dy$


Note that 

$$
\begin{aligned}
\mathcal{L}_u (s) &= \mathcal{L}_F (s) + \mathcal{L}_u (s) \cdot \mathcal{L}_p (s) \\[8pt]
&= \frac{\mathcal{L}_p(s)}{1-\mathcal{L}_p(s)}
\end{aligned}
$$


So we can follow 

\begin{itemize}
	\item $F \to \mathcal{L}_p$
	\item $\mathcal{L}_p \to \mathcal{L}_u$
	\item $\mathcal{L}_u \to u$
\end{itemize}

where the inverse Laplace transform can be obtained using Bromwich integral



For example, let $\{S_n \}_{n=1}^\infty$ to be

$$
S_n := S_{n-1} + \xi_n
$$

where $\xi_i \sim p(x) = e^{-x}/2 + e^{-2x}, x>0$. Then $\mathbb{E} (N_t^{-})?$


\textbf{$p \to \mathcal{L}(p)$}. 

$$
\begin{aligned}
\mathcal{L}(p) = \frac{1}{2(s+1)} + \frac{1}{s+2} = \frac{3s + 4}{2(s+1)(s+2)}  
\end{aligned}
$$


\subsection{Limit theorems for renewal processes}


Consider a stochastic process $\{S_n \}_{n=1}^\infty$,

$$
S_n := S_{n-1} + \xi_n
$$

for IID $\xi_i > 0$. 

\begin{theorem}
\textbf{Theorem 1}. Assume $\mu := \mathbb{E} (\xi_1) < \infty$. Then 

$$
N_t / t \overset{t \to \infty}{\to} 1/\mu, a.s.
$$

This is analogous to \textbf{SLLN}, where 

$$
(\xi_1 + \cdots + \xi_n) / n \overset{t \to \infty}{\to}  \mu, a.s.
$$
\end{theorem}


\textbf{Proof} 

$$
\begin{aligned}
&S_{N_t} \le t \le S_{N_{t+1}} \\[8pt]
&\frac{N_t}{S_{N_{t+1}}} \le \frac{N_t}{t} \le \frac{N_t}{S_{N_t}} 
\end{aligned}
$$

where we have

$$
\begin{aligned}
&\lim_{t\to\infty} \frac{N_t}{S_{N_t}} = \lim_{n\to\infty} \frac{n}{S_n} = 1/\mu \\[8pt]
&\lim_{t\to\infty} \frac{N_t}{S_{N_{t+1}}} = 
\lim_{t\to\infty} \frac{N_t}{N_{t+1}} \frac{N_{t+1}}{S_{N_{t+1}}} = 1/\mu
\end{aligned}
$$


\begin{theorem}
\textbf{Theorem 2}. Assume $\sigma^2 = \mathrm{Var} (\xi_1) < \infty$. Then

$$
Z_t := \frac{N_t - t/\mu}{\sigma \sqrt{t} / \mu^{3/2}} \overset{d}{\to} N(0,1)
$$

This is analogous to \textbf{CLT}, where 

$$
\frac{\xi_1 + \cdots + \xi_n - n\mu}{\sigma \sqrt{n}} \overset{d}{\to}  N(0,1)
$$
\end{theorem}


\textbf{Proof} 

Note that 

$$
P \left( \frac{S_n - n\mu}{\sigma \sqrt{n}} \le x \right) \to P(x)
$$


\pagebreak
%----------------------------------------------------------------------------------------
%	Appendix
%----------------------------------------------------------------------------------------
\section*{Karlin: Renewal processes}
\setcounter{section}{0}

\section{Renewal Process: Definition and Concepts}

A \textbf{renewal (counting) process} $\{ N(t) : t \ge 0 \}$ is a nonnegative integer-valued stochastic process that registers the successive 

cf. A \textbf{counting process} is a process $\{ N(t) : t \ge 0 \}$ such that

\begin{itemize}
	\item $N(t) \ge 0$
	\item $N(t)$ is an integer
	\item $s \le t$, then $N(s) \le N(t)$
\end{itemize}

\section{Examples}

\subsection{Poisson Processes}

A Poisson process $\{N(t): t \ge 0 \}$ with $\lambda$ is a renewal counting process having the exponential interoccurrence distribution

$$
F(x) = 1 - e^{-\lambda x}, x \ge 0
$$

Note that 

$$
P(W_r > t) = P(N_t \le r-1)
$$

where $W_r$ is the time taken for $r$th event, and $N_t$ is the number of events that cumulated until time $t$. By integral by parts, we can obtain

$$
\int_t^\infty \frac{\lambda^r y^{r-1} exp(-\lambda y)}{\Gamma(r)} dy = \sum_{k=0}^{r-1} \frac{exp(-\lambda t (\lambda t)^k)}{k!}
$$



\end{document}