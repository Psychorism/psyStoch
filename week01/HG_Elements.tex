\documentclass[12pt]{article} 

\usepackage{geometry}
\geometry{a4paper} 

\usepackage{graphicx} 
\usepackage{enumitem}
\usepackage{booktabs}

\usepackage{float} 
\usepackage{wrapfig} 

\usepackage{amsmath}
\usepackage{amsfonts}
\usepackage{amssymb}
\usepackage{dsfont}

\usepackage{xcolor}
\usepackage{listings}
\usepackage{caption}
\DeclareCaptionFont{white}{\color{white}}
\DeclareCaptionFormat{listing}{%
  \parbox{\textwidth}{\colorbox{gray}{\parbox{\textwidth}{#1#2#3}}\vskip-2pt}}
\captionsetup[lstlisting]{format=listing,labelfont=white,textfont=white}
\lstset{frame=lrb,xleftmargin=\fboxsep,xrightmargin=-\fboxsep}

\linespread{1.2} 
\setlength{\parskip}{\baselineskip} % vertical spaces
\setlength\parindent{0pt} % remove all indentation from paragraphs


\usepackage{ntheorem}
\usepackage{mdframed}

\theoremstyle{nonumberbreak}
\theoremheaderfont{\bfseries}
\newmdtheoremenv[%
linecolor=gray,leftmargin=10,%
rightmargin=10,
backgroundcolor=gray!20,%
innertopmargin=0pt,%
ntheorem]{theorem}{}




\begin{document}

\title{\textbf{Supplementary: \\ Elements of Stochastic Processes}}
\author{Hyunwoo Gu}
\date{}

\maketitle


%----------------------------------------------------------------------------------------
%   Section 0
%----------------------------------------------------------------------------------------

\section{Basic terminology}

\textbf{Lebesgue-Stieltjes integration}

A generaliation of Riemann-Stieltjes integration and Lebesgue integration, which is just the ordinary Lebesgue integral with respect to a measure known as the \textbf{Lebesgue–Stieltjes measure}. We have

\begin{center}

\end{center}


The Lebesgue-Stieltjes integration

$$
\int_a^b f(x) d g(x)
$$

is defined when $f: [a,b] \to \mathbb{R}$ is \textbf{Borel-measurable} and \textbf{bounded}, and $g : [a,b] \to \mathbb{R}$ is of bounded variation



\textbf{Radon-Nikodym theorem}

If 


\textbf{Expected value}

$$
\mathbb{E}[g(X)] = \int g(x) dF_X(x)
$$


\textbf{Convolution}

$$
\mathbb{E}[g(X)] = \int g(x) dF_X(x)
$$




\textbf{Problems}






\end{document}