\documentclass[12pt]{article} 

\usepackage{geometry}
\geometry{a4paper} 

\usepackage{graphicx} 
\usepackage{enumitem}
\usepackage{booktabs}

\usepackage{float} 
\usepackage{wrapfig} 

\usepackage{amsmath}
\usepackage{amsfonts}
\usepackage{amssymb}
\usepackage{dsfont}

\usepackage{xcolor}
\usepackage{listings}
\usepackage{caption}
\DeclareCaptionFont{white}{\color{white}}
\DeclareCaptionFormat{listing}{%
  \parbox{\textwidth}{\colorbox{gray}{\parbox{\textwidth}{#1#2#3}}\vskip-2pt}}
\captionsetup[lstlisting]{format=listing,labelfont=white,textfont=white}
\lstset{frame=lrb,xleftmargin=\fboxsep,xrightmargin=-\fboxsep}

\linespread{1.2} 
\setlength{\parskip}{\baselineskip} % vertical spaces
\setlength\parindent{0pt} % remove all indentation from paragraphs

\usepackage{multirow}
\usepackage{ntheorem}
\usepackage{mdframed}

\theoremstyle{nonumberbreak}
\theoremheaderfont{\bfseries}
\newmdtheoremenv[%
linecolor=gray,leftmargin=10,%
rightmargin=10,
backgroundcolor=gray!20,%
innertopmargin=0pt,%
ntheorem]{theorem}{}




\begin{document}

\title{\textbf{Levy processes}}
\author{Hyunwoo Gu}
\date{}

\maketitle


%----------------------------------------------------------------------------------------
%   Section 1
%----------------------------------------------------------------------------------------
\section{Introduction to the theory of Levy processes}

Here we will be able to: 

\begin{itemize}
	\item understand the main properties of Levy processes
	\item construct ta Levy process from an infinitely 
\end{itemize}


\subsection{Definition of a Levy process}

Levy processes are widely used for modelings of jump dynamics. 

\begin{center}
	\begin{tabular}{ |c|c|c| } 
		\hline
		$N_t$ & $W_t$ & Levy \\
		\hline
		$N_0=0$ & $W_0=0$ & $L_0=0$ \\ 
		inpt. incr. & inpt. incr. & inpt. incr. \\ 
		sta. incr. & sta. incr. & sta. incr. \\ 
		$N_t-N_s \sim Pois(\lambda (t-s))$ & 
		$W_t-W_s \sim N(0,t-s)$ &
		$L_t-L_s \sim P(t-s)$ \\
		\hline
	\end{tabular}
\end{center}

where $P(\cdot)$: \textbf{infinite divisible} distribution. 


\subsection{Stochastic continuity and cadlag paths}

$L_t$ is \textbf{stochastically continuous} if $L_{t+h} \overset{P,h\to 0}{\to} L_t$, or $\forall \epsilon >0$

$$
P [ | L_{t+h} - L_t] > \epsilon ] \overset{h\to0}{\to} 0
$$

The sample path of \textbf{Levy process} follows a cadlag path. 


\subsection{Examples of Levy processes}

The class of \textbf{infinitely divisible distributions}. 




\begin{theorem}
\textbf{Definition}. An RV $\xi$ is a \textbf{infinitely divisible distribution}, if 

$$
\xi \overset{d}{\equiv} Y_1 \oplus \cdots \oplus Y_n
$$

\end{theorem}


It is not true that the distribution is stable iff it is infinitely divisible. 




\textbf{Example 1}. 


\textbf{Example 2}. 


\subsection{Characteristic Exponent}



\begin{theorem}
\textbf{Proposition}. For $\forall$ Levy process $L_t$

$\exists \psi: \mathbb{R} \to \mathbb{C}$ such that

$$
\phi_{L_t} (u) = \mathbb{E} \left[ e^{iuL_t} \right] = e^{t \psi(u)}
$$

\end{theorem}

\subsection{Properties of Levy Characteristic Exponent}

\textbf{Levy measure}. 


\begin{theorem}
\textbf{Definition}. 

$$
\nu (B) = \mathbb{E} [\# t \in [0,1] : \Delta X_t \in B]
$$

for any $B \subset \mathbb{R} - \{ 0 \}$
\end{theorem}


\subsection{Levy-Khintchine theorem}


$$
\phi_{X_t}(u) = exp \left[ t (iu\mu - \sigma^2 u^2 /2) + \int_\mathbb{R} \left( e^{iux} + 1 - iux \mathbf{1}_{|x| < 1} \right) \nu (dx) \right] 
$$

\textbf{Example 1}. For $X_t$ : of bounded variation,

$$
\sum_{k=1}^n | X_{t_k} - X_{t_{k-1}} |  \overset{max |t_i - t_{i-1}| \to 0}{\longrightarrow} < \infty
$$

Note that the Brownian motion is not a process of bounded variation. $X_t$ is of bounded variation iff

$$
\sigma=0, \int_{|x|<1} x\nu (dx) < \infty
$$

By letting $\sigma \equiv 0$, we have

$$
\phi_{X_t}(u) = exp\left\{ t \left( iu \tilde{\mu} + \int(e^{iux}-1)\nu(dx) \right) \right\}
$$

where $\tilde{\mu} := \mu - \int_{|x|<1} x d\nu(dx)$. 


\textbf{Example 2}. For $X_t$ : \textbf{compound Poisson process}, 

$$
\sigma = 0, \int_\mathbb{R} \nu(dx) = \nu(\mathbb{R}) < \infty
$$


\textbf{Example 3}. For $X_t$ : \textbf{Subordinators}, 

$$
X_t \ge 0 \ \ a.s. \Leftrightarrow X_t \ge X_s \ \ a.s.
$$




\textbf{Properties}. 

We have the following property:

$$
\int_{|x| < 1} x^2 \nu (dx) < \infty, \int_{|x| > 1} \nu (dx) < \infty
$$

$$
J := \int_{|x|<1} \left( e^{iux} -1 -0  \right)
$$



\subsection{Modeling of jump-type dynamics}

How to estimate that the Levy measure of a Levy process from the following data? 

$$
X_t: X_\Delta
$$


CIR model for the stochastic volatility is 

$$
c \sqrt{V_t} d W_t
$$


Levy process can be used to model the properties of jobs. 


Let $X_t$ be of bounded variation, and 

$$
\phi_{X_\Delta} (u) := exp \left\{  \Delta \left( iu\mu + \int_\mathbb{R} (e^{iux} - 1) S(x) dx \right)  \right\}
$$

Note that 

$$
\phi_{X_\Delta} (u) := exp \left\{  \Delta \left( iu\mu + \int_\mathbb{R} (e^{iux} - 1) S(x) dx \right)  \right\}
$$


Let $X_t$ be a Levy process of bounded variation with Levy triplet $(\mu, \sigma^2, \nu)$ and Levy density $s(x)$. The correct form of the characteristic exponent is given as

$$
\psi (u) = iu \left( \mu - \int_{|x| < 1} x s(x) dx  \right) + \int_\mathbb{R} (e^{iux} - 1) s(x) dx
$$

with $\mu$ probably equal to zero. Note that it is also true that nonzero $\mu$ can be of bounded variation. 



\subsection{Time-changed stochastic processes. Monroe theorem}

The following are the \textbf{stylized facts} of financial data. 

\textbf{1. Stochastic time change} 

For $\{ X_t \}$: Levy process, and a subordinator $T(s)$, and thus $X_{T(s)}$. The \textbf{Monroe's theorem} states the equivalence of the set of $W_{T(s)}$ types and the set of submartingales, where $W, T$ are possibly dependent. 

Assume that $W \perp T$. Note that 

$$
\phi_{X_{T(s)}} (u) = \mathcal{L}_{T(s)} \left( - \psi(u) \right)
$$

where $\psi(u)$ is the characteristic exponent of process $X$. 



\textbf{2. Stochastic volatility}

For \textbf{Black-Scholes},

$$
d(ln S_t) = \left( \mu - \sigma^2/2 \right)dt + \sigma dW_t
$$ 

where the volatility parameter $\sigma \to V_t \ge 0$. For example, \textbf{Cox-Ingersoll-Ross} is defined as the SDE

$$
dV_t = (a - bV_t) dt + c \sqrt{V_t} d W_t
$$




\pagebreak
\subsection*{Quizzes}



\textbf{(Quiz 1-3)}. $X_t := bt + \sigma W_t + c N_t$, where $W_t$: Brownian motion, $N_t$: a Poisson process with $\lambda$, and $W_t, N_t$ independent, $b,c \in \mathbb{R}$, $\sigma \ge 0$. Denote the Levy measure of this process bt $\nu$.  


\textbf{(Quiz 1)}. Find the characteristic function of this process.

\textbf{(Answer)}. $exp \{ iubt + \lambda t (e^{icu} - 1) - \frac{t(\sigma u)^2}{2} \}$


\textbf{(Quiz 2)}. What is measure $\nu$ of a Borel set $B$? 

\textbf{(Answer)}. $\nu(B) = \lambda$, if $1 \in B$ and $0$ otherwise.

Considering that the jump part of the Levy process discussed is

\begin{itemize}
	\item $c$: real number
	\item $N_t$ : Poisson process with $\lambda$
\end{itemize}

Thus $\nu(B) = \lambda \mathbf{1}_{c \in B}$. Computing the integral in the characteristic exponent, returns a pointwise evaluation in the point $c$, $\lambda (e^{iuc} - 1)$.


\textbf{(Quiz 3)}. Let $X_t$ be a Levy process. What is the correct expression for $\mathrm{Var} (X_t)$ in terms of characteristic exponent $\psi$? 

\textbf{(Answer)}. $\mathrm{Var} (X_t) = -t \psi''(0)$


\textbf{(Quiz 4)}. Let $X_t$ be a Levy process, assuming $X_1 \sim N(0,1)$. Find the mean and the variance of $X_t$

\textbf{(Answer)}. It is just the Brownian motion. Therefore, $\mathbb{E} [X_t] = 0, \mathrm{Var} (X_t) = t$.


\textbf{(Quiz 5)}. Let $X_t = bt + N_t$, where $N_t$ is a Poisson process with $\lambda$ and $b \in \mathbb{R}$. Find the Levy triplet of this process. 

\textbf{(Answer)}. $(b + \lambda, 0, \nu)$, where $\nu(B) =\lambda \mathbf{1}_{1 \in B}$ for any Borel set $B$.

The first term of a Levy triplet should correspond to the coefficient of the drift term, i.e. $b$, if $X$ is of the form

$$
X_t = bt + \sigma W_t + c N_t
$$



\end{document}