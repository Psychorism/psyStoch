\documentclass[12pt]{article} 

\usepackage{geometry}
\geometry{a4paper} 

\usepackage{graphicx} 
\usepackage{enumitem}
\usepackage{booktabs}

\usepackage{float} 
\usepackage{wrapfig} 

\usepackage{amsmath}
\usepackage{amsfonts}
\usepackage{amssymb}
\usepackage{dsfont}

\usepackage{xcolor}
\usepackage{listings}
\usepackage{caption}
\DeclareCaptionFont{white}{\color{white}}
\DeclareCaptionFormat{listing}{%
  \parbox{\textwidth}{\colorbox{gray}{\parbox{\textwidth}{#1#2#3}}\vskip-2pt}}
\captionsetup[lstlisting]{format=listing,labelfont=white,textfont=white}
\lstset{frame=lrb,xleftmargin=\fboxsep,xrightmargin=-\fboxsep}

\linespread{1.2} 
\setlength{\parskip}{\baselineskip} % vertical spaces
\setlength\parindent{0pt} % remove all indentation from paragraphs


\usepackage{ntheorem}
\usepackage{mdframed}

\theoremstyle{nonumberbreak}
\theoremheaderfont{\bfseries}
\newmdtheoremenv[%
linecolor=gray,leftmargin=10,%
rightmargin=10,
backgroundcolor=gray!20,%
innertopmargin=0pt,%
ntheorem]{theorem}{}




\begin{document}

\title{\textbf{Birth and Death Processes(Sup)}}
\author{Hyunwoo Gu}
\date{}

\maketitle

%----------------------------------------------------------------------------------------
%	Supplementary: Birth and Death Processes
%----------------------------------------------------------------------------------------
\section*{Birth-death Markov Chains}(6.4. from \textit{Gallager})

A \textit{birth-death Markov chain} is a Markov chain where the state space is the set of nonnegative integers. A transition from $i$ to $i+1$ is regarded as a birth and $i+1$ to $i$ is regarded as a death. Let us denote $p_i := P_{i,i+1}$ and $q_i := P_{i,i-1}$, thus $P_{ii} = 1 - p_i - q_i$. So what is the steady-state probabilities of these BD chains?

Note that the number of transitions from $i$ to $i+1$ differs by at most $1$ from the number of transitions 


Thus if we visualize renewal-reward process with renewals on occurrences of state $i$ and unit reward on transitions from $i$ to $i+1$, the limiting time-average number of transitions per unit time is $\pi_i p_i$. Similarly, the limiting time-average number of transitions per unit time from $i+1$ to $i$ is $\pi_{i+1} q_{i+1}$. Thus by equating in limit,

$$
\pi_i p_i = \pi_{i+1} q_{i+1}, \ \ \forall i\ge 0
$$


It is convenient to define $\rho_i$ as $p_i / q_{i+1}$. Then we have $\pi_{i+1} = \rho_i \pi_i$, and iterating this,

$$
\pi_i = \pi_0 \prod_{j=0}^{i-1} \rho_j, \ \ \pi_0 := \frac{1}{1 + \sum_{i=1}^\infty \prod_{j=0}^{i-1} \rho_j}
$$

If $\sum_{i=1}^\infty \prod_{j=0}^{i-1} \rho_j < \infty$, then $\pi_0$ is positive and \textbf{all the states are positive recurrent}. If this sum of products is infinite, then no state is positive recurrent. 


\section{General pure birth processes}

In this chapter we present a brief discussion of several important examples of continuous time discrete state Markov processes with the stationary transition probabilities such that 

$$
P_{ij}(t) = Pr\{ X(t+u) = j | X(u) = i \}
$$

for $t>0, i,j=1,2,\cdots$, independent of $u \ge 0$. 

\subsection{Postulates for the Poisson process}

In order to define more general processes of a similar kind, let us point out various further properties that the Poisson possesses. In particular, it is a Markov process on the nonnegative integers which has the following properties :


$$
\begin{aligned}
P\{ X(t+h) - X(t) = 1 | X(t) = x \} &= \lambda h + o(h), h \to 0_+ \\[10pt]
\Leftrightarrow \mathrm{lim}_{h \to 0+} \frac{P\{ X(t+h) - X(t) = 1 | X(t) = x \}}{h} &= \lambda
\end{aligned}
$$

where $x=0,1,2,\cdots$. Notice that the right-hand side is independent of $x$. 


$$
P\{ X(t+h) - X(t) = 0 | X(t) = x \} = 1 - \lambda h + o(h), h \to 0_+
$$

$$
X(0) = 0 
$$


\subsection{Pure birth process}

A natural generalization of the Poisson process is to permit the chance of an event occurring at a given instant of time to depend upon the number of events which have already occurred. For example, \textbf{the probability of a birth at a given instant is proportional to the population size at that time}, which is known as the \textbf{Yule process}.


The characteristic function of $S_n$ is given by

$$
\phi_n(w) = \mathbb{E}(exp(iwS_n)) = \prod_{k=0}^{n-1} \mathbb{E}(exp(iw T_k)) = \prod_{k=0}^{n-1} \frac{\lambda_k}{\lambda_k - iw}
$$


\subsection{Yule process}

The \textbf{Yule process} is an example of a \textbf{pure birth process} that arises in physics and biology. Assume that each member in a population has a probability $\beta h + o(h)$ of giving birth to a new member in an interval of time length $h (\beta > 0)$. Furthermore assume that there are $X(O) = N$ members present at time $0$. Assuming independence and no interaction among members of the population, the binomial theorem gives

$$
\begin{aligned}
Pr[X(t+h) - X(t) = 1 | X(t) = n] &= \binom{n}{1} \left[ \beta h + o(h) \right] \left( 1 - \beta h + o(h) \right)^{n-1} \\[8pt]
&= n\beta h + o_n(h)
\end{aligned}
$$

where $\lambda_n = n \beta$. Thus

$$
P_n'(t) = -\beta [nP_n(t) - (n-1) P_{n-1}(t) ], \ \ n=1,2,\cdots
$$

under BV : $P_1(0)=1, P_n(0)=0, \ \ n=2,3,\cdots$. 

The solution is 

$$
P_n(t) = e^{-\beta t} (1 - e^{-\beta t})^{n-1}
$$

The generating function is 

$$
\begin{aligned}
f(x) &= \sum_{n=1}^\infty P_n(t) s^n \\[8pt]
&= s e^{-\beta t} \sum_{n=1}^\infty \left[ (1-e^{-\beta t}) s \right]^{n-1} \\[8pt]
&= \frac{se^{-\beta t}}{ (1 - (1-e^{-\beta t})s) }
\end{aligned}
$$

Letting $P_{N_n}(t) := Pr[X(t) = n | X(0) = N]$ and $f_N(s) = \sum_{n=N}^\infty P_{N_n}(t)s^n$, we have


$$
\begin{aligned}
f_N(s) &= [f(s)]^N \\[8pt]
&= \left[ \frac{se^{-\beta t}}{1 - (1-e^{-\beta t})s} \right]^N \\[8pt]
&= (s e^{-\beta t})^N \sum_{m=0}^infty \binom{m+N-1}{m} (1-e^{-\beta t})^m s^m
\end{aligned}
$$



$T$ : the waiting time of $X(t)$ in the state $i$. 


Letting $G_i(t) := P(T_i \ge t)$, 

$$
\begin{aligned}
G_i (t+h) &= G_i(t) G_i(h) = G_i(t) \left[ P_{ii}(h) + o(h) \right] \\[8pt]
&= G_i(t) [ 1- (\lambda_i + \mu_i)h ] + o(h) \\[10pt]
\Leftrightarrow \frac{ G_i(t+h) - G_i(t)  }{ h } = -(\lambda_i + \mu_i) G_i(t) + o(1) \\[8pt]
G_i'(t) &= -(\lambda_i + \mu_i) G_i(t)
\end{aligned}
$$

where the last line corresponds to the IVP of with $G_i(0) = 1$


\section{More about Poisson Processes}


\textbf{Theorem 2.2.} If $F(x)$ is a distribution such that $F(0)=0$ and $F(x)<1$ for some $x>0$, then $F(x)$ is an exponential distribution if and only if 

$$
F(x+y) - F(y) = F(x) \left[ 1 - F(y) \right]
$$

for all $x,y \ge 0$.

\textbf{Proof}.


\section{A Counter Model}

Consider the following problem : Electrical pulses with random amplitudes
(i.e., according to a Poisson process) at a detector whose output for each pulse at time tis


\section{Birth and Death Processes}

To generalize the pure birth processes, we can permit $X(t)$ to decrease as well as increase, for example, by the death of members. This can be regarded as the continuous time analogs of random walks. 


\begin{itemize}
	\item $P_{i, i+1}(h) = \lambda_i h + o(h), h \to 0_+$, $i \ge 0$
	\item $P_{i, i-1}(h) = \mu_i h + o(h), h \to 0_+$, $i \ge 1$
	\item $P_{i,I} (h) = 1 -(\lambda_i + \mu_i) h + o(h), h \to 0_+$, $i \ge 0$
	\item $P_{ij} (0) = \delta_{ij}$
	\item $\mu_0 = 0$, $\lambda_0 > 0, \mu_i, \lambda_i > 0, i=1,2,\cdots$
\end{itemize}

The matrix $A$, the \textbf{infinitesimal generator} of the process,

$$
A = \begin{bmatrix}
-\lambda_0 & \lambda_0 & 0 & 0 & \cdots \\
\mu_1 & -(\lambda_1 + \mu_1)  & \lambda_1 & 0 & \cdots \\
0 & \mu_2 & -(\lambda_2 + \mu_2) & \lambda_2 & \cdots \\
0 & 0 & \mu_3 & -(\lambda_3 + \mu_3) & \cdots
\vdots & \vdots &  \vdots &  \vdots &   & 
\end{bmatrix}
$$


\section{Differential Equations of Birth and Death Processes}

As in the pure birth and Poisson processes, the transition probabilities $P_{ij}$ satisfy a system of differential equations, known as \textit{Kolmogorov differential equations}. 

$$
\begin{aligned}
P_{0j}'(t) &= -\lambda_0 P_{0j} (t) + \lambda_0 P_{ij}(t) \\[8pt]
P_{ij}'(t) &= \mu_i P_{i-1, j} (t) - (\lambda_i + \mu_i) P_{ij}(t) + \lambda_i P_{i+1, j}(t) \\[8pt]
\end{aligned}
$$

where the boundary condition $P_{ij}(0) = \delta_{ij}$. 



\section{Examples of Birth and Death Processes}

\subsection{Linear Growth with Immigration}

A birth and death process is called a \textbf{linear growth process} if $\lambda_n = \lambda_n + a$ and $\mu_n = \mu n$ with $\lambda, \mu, a >0$. Such 



\subsection{Queueing}

A \textbf{queueing process} is a process in which customers arrive at some place where a service is being rendered. It is assumed that interarrival time and the service for a given customer are governed by probabilistic laws. The length of the queue at a given time $t$ is represented by $X(t)$. 

If we let $\lambda_i = \lambda$ for all $i$ in the general birth and death process, the resulting process is a special simple case of a \textbf{continuous time queueing process}. 

The state of the system is then interpreted as the length of a queue where the times between arrivals of the customers are independent random variables with an exponential distribution of parameter $\lambda$ and for which the duration of the service time of the current customer is a random variable with an exponential distribution whose parameter, $\mu_n$, may depend on the length of the line. At the completion of each service the line decreases by $1$ and with each new arrival the line increases by $1$ . The
classical case of a single-server queue corresponds to J.li === J.l., i > 1 ., i.e . .,
each service follows the same exponential distribution with parameter J.1 independent of the length of the waiting line.
The classical telephone trunking model can be formulated as a queueing
birth and death process with infinitely many servers., each of whose service
time distribution has the same parameter J.l., so that Jli === iJ.l., i > 1 . The
rationale underlying this specification goes as follows : Suppose the queue
consists of i individual customers ; then since the number of servers is
unlimited each customer is simultaneously receiving service. Now the
length of service of each is independent of the others and distributed
exponentially with parameter J.l· It follows that the probability distribution
of the time until at least one of the customers completes service
(i.e . ., the length of time until the waiting line decreases by 1) is also exponentially
distributed., but is now of parameter i11 (the student should
prove this) .

Besides the two special cases mentioned above it is possible to consider
numerous other queueing - models by appropriate specifications of the
parameters Ilk . For example., a queue with n servers., each of whose service
time has an exponential distribution with the same parameter J.l., would
correspond to Ilk === k11 for 1 < k < n., fli === n11 for i > n.


For the single-server process with $\lambda < \mu$ the stationary distribution is 


\section{Birth and Death Processes with Absorbing States}

\textbf{Probability of absorption into state $0$}


$$
\mathbf{u}_i = \frac{\lambda_i}{\mu_i + \lambda_i} \mathbf{u}_{i+1}  + \frac{\mu_i}{\mu_i + \lambda_i} \mathbf{u}_{i-1}
$$

where $u_0 = 1$. 


\textbf{Probability of absorption into state $0$}



\begin{theorem}
\textbf{Theorem 7.1.}. Consider a BDP with birth and death parameters $\lambda_n$ and $\mu_n$ with $n \ge 1$, where $\lambda_0 = 0$ so that $0$ is an absorbing state. 

\textbf{The probability of absorption into state $0$ from the intial state $m$} is given as 

$$
\begin{cases}
\frac{ \sum_{i=m}^\infty \left( \prod_{j=1}^i \mu_j/\lambda_j \right)  }{ 1+ \sum_{i=1}^\infty \left( \prod_{j=1}^i \mu_j/\lambda_j \right)} & \sum_{i=1}^\infty \left( \prod_{j=1}^i \mu_j/\lambda_j \right) < \infty \\
1 & \sum_{i=1}^\infty \left( \prod_{j=1}^i \mu_j/\lambda_j \right) = \infty
\end{cases}
$$

The \textbf{mean time to absorption} is



\end{theorem}

\textbf{Proof}. ($\Rightarrow$) 



\section{Finite State Continuous Time Markov Chains}






For the single-server process with $\lambda < \mu$ the stationary distribution is 

$$
\pi_n = \frac{\lambda_0 \lambda_1 \cdots \lambda_{n-1} }{\mu_1 \mu_2 \cdots \mu_n} = (\frac{\lambda}{\mu})^n
$$

which, when normalized, results in

$$
P_n = \frac{\mu - \lambda}{\mu} (\frac{\lambda}{\mu})^n, \ \ n \ge 0
$$


If the process has been going on a long time and $\lambda < \mu$, the probability of being served immediately upon arrival is 

$$
P_0 = (1 - \frac{\lambda}{\mu})
$$


If an arriving customer finds $n$ people in front of her, her total waiting time $T$, including his own service time, is the sum of service times of herself and those ahead, all distributed exponentially with param $\mu$, thus 

$$
\begin{aligned}
T | n \ \ \mathrm{ahead} &\equiv Gamma(n+1, \mu) \\[8pt]
\Leftrightarrow \ \ P \{ T \le t | n \ \ \mathrm{ahead}  \} &= \int_0^t \frac{\mu^{n+1} \tau^n e^{-\mu t} }{\Gamma(n+1)} d\tau \\[10pt]
\therefore \ \ P \{ T \le t\} &= \sum_{n=0}^\infty P \{ T \le t | n \ \ \mathrm{ahead} \} \cdot (\frac{\lambda}{\mu})^n (1 - \frac{\lambda}{\mu})
\end{aligned}
$$  

since $(\frac{\lambda}{\mu})^n (1 - \frac{\lambda}{\mu})$ is the probability that in the stationary case a customer on arrival will find $n$ ahead in line. 


$$
\begin{aligned}
P \{ T \le t\} &= \\[8pt]
\end{aligned}
$$



$$
\begin{aligned}
M(t) &= \sum_{j=0}^\infty j P_{ij}(t) \\[8pt]
M'(t) &= \lambda - \mu M(t) \\[8pt]
M(t) &= \frac{\lambda}{\mu} (1 - e^{-\mu t}) + i e^{-\mu t} \\[8pt]
\end{aligned}
$$

If we let $t \to \infty$, then $M(t) \to \lambda/\mu$, which is the mean value of the stationary distribution given above. 







\end{document}