\documentclass[12pt]{article} 

\usepackage{geometry}
\geometry{a4paper} 

\usepackage{graphicx} 
\usepackage{enumitem}
\usepackage{booktabs}

\usepackage{float} 
\usepackage{wrapfig} 

\usepackage{amsmath}
\usepackage{amsfonts}
\usepackage{amssymb}
\usepackage{dsfont}

\usepackage{xcolor}
\usepackage{listings}
\usepackage{caption}
\DeclareCaptionFont{white}{\color{white}}
\DeclareCaptionFormat{listing}{%
  \parbox{\textwidth}{\colorbox{gray}{\parbox{\textwidth}{#1#2#3}}\vskip-2pt}}
\captionsetup[lstlisting]{format=listing,labelfont=white,textfont=white}
\lstset{frame=lrb,xleftmargin=\fboxsep,xrightmargin=-\fboxsep}

\linespread{1.2} 
\setlength{\parskip}{\baselineskip} % vertical spaces
\setlength\parindent{0pt} % remove all indentation from paragraphs


\usepackage{ntheorem}
\usepackage{mdframed}

\theoremstyle{nonumberbreak}
\theoremheaderfont{\bfseries}
\newmdtheoremenv[%
linecolor=gray,leftmargin=10,%
rightmargin=10,
backgroundcolor=gray!20,%
innertopmargin=0pt,%
ntheorem]{theorem}{}




\begin{document}

\title{\textbf{Stochastic Integration \\ \& Ito Formula}}
\author{Hyunwoo Gu}
\date{}

\maketitle


%----------------------------------------------------------------------------------------
%   Section 1
%----------------------------------------------------------------------------------------
\section{Stochastic Integration}

\subsection{Different types of stochastic integrals}

\subsection{Integrals of the type $\int X_t dt$}


\subsection{Integrals of the type $\int f(t) d W_t$}

\subsection{Integrals of the type $\int X_t d W_t$}



\pagebreak
%----------------------------------------------------------------------------------------
%   Section 2
%----------------------------------------------------------------------------------------
\section{Ito Formula}


\subsection{Integrals of the type $\int X_t d Y_t$}

$$
\int_a^l X_t dH_t
$$

where $H_t$ : Ito process. 



\subsection{Ito's formula}



\subsection{Ornstein-Uhlenbeck process}




\subsection*{Quizzes}


\textbf{(Quiz 1)}. Let $X_t := cos(wt + \theta)$ be a stochastic process and $\theta \sim Unif(0,2\pi)$, with $w=\pi/10$. Classify this process.

\textbf{(Answer)} \textbf{Ergodic} and \textbf{weak stationary}. 


\textbf{(Quiz 2)}. Let $X_t := \epsilon_t + \xi cos(\pi t/12)$, $t=1,2,\cdots$, where $\xi, \epsilon_1, \epsilon_2, \cdots$ are IID standard normal random variables.

\textbf{(Answer)} Not \textbf{weak stationry}, but \textbf{ergodic}.


\textbf{(Quiz 3)}. Assume that for a process $X_t$ it is known that $\mathbb{E} (X_t) = \alpha + \beta t$, $cov(X_t, X_{t+h} = e^{-h \lambda}$ for all $h \ge 0, t >0$, and some constants $\lambda >0, \alpha, \beta$. Classify the process $Y_t := X_{t+1} - X_t$. 

\textbf{(Answer)} $Y_t$ is weakly stationary and ergodic.


\textbf{(Quiz 4)}. Let $X_t := \sigma W_t + ct$, where $W_t$ is Brownian motion, $\sigma, c >0$. Choose the correct statements about this process. 

\textbf{(Answer)} $X_t$ has \textbf{continuous trajectories}. 


\textbf{(Quiz 5)}. Let $X_t$ have an autocovariance function $\gamma(r) := e^{-\alpha |r|}$. Is $Y_t := X_t + w$ an ergodic process?


\textbf{(Answer)} Yes, if $w$ is a constant. 



\end{document}