\documentclass[12pt]{article} 

\usepackage{geometry}
\geometry{a4paper} 

\usepackage{graphicx} 
\usepackage{enumitem}
\usepackage{booktabs}

\usepackage{float} 
\usepackage{wrapfig} 

\usepackage{amsmath}
\usepackage{amsfonts}
\usepackage{amssymb}
\usepackage{dsfont}

\usepackage{xcolor}
\usepackage{listings}
\usepackage{caption}
\DeclareCaptionFont{white}{\color{white}}
\DeclareCaptionFormat{listing}{%
  \parbox{\textwidth}{\colorbox{gray}{\parbox{\textwidth}{#1#2#3}}\vskip-2pt}}
\captionsetup[lstlisting]{format=listing,labelfont=white,textfont=white}
\lstset{frame=lrb,xleftmargin=\fboxsep,xrightmargin=-\fboxsep}

\linespread{1.2} 
\setlength{\parskip}{\baselineskip} % vertical spaces
\setlength\parindent{0pt} % remove all indentation from paragraphs


\usepackage{ntheorem}
\usepackage{mdframed}

\theoremstyle{nonumberbreak}
\theoremheaderfont{\bfseries}
\newmdtheoremenv[%
linecolor=gray,leftmargin=10,%
rightmargin=10,
backgroundcolor=gray!20,%
innertopmargin=0pt,%
ntheorem]{theorem}{}




\begin{document}

\title{\textbf{Fourier Transform(sup)}}
\author{Hyunwoo Gu}
\date{}

\maketitle

\subsection*{Ideas of Derivation}

Suppose $f(x)$ : a periodic of $2\pi$

$$
\begin{aligned}
\hat{f}(n) &= \int_{-\infty}^\infty f(y) exp(-iny) \\[8pt]
\end{aligned}
$$


\subsection*{Properties}

\begin{theorem}
\textbf{Prancherel's theorem}. 

$$
\int_{-\infty}^\infty \vert f(t) \vert^2 dt = \frac{1}{2\pi} \int_{-\infty}^\infty \vert \hat{f}(\omega) \vert^2 d\omega
$$

\end{theorem}

\textbf{Proof}. Note that 

$$
\begin{aligned}
 &= \\[8pt]
\end{aligned}
$$




\begin{theorem}
\textbf{Theorem}. For $f$: $2\pi$-periodic with $f^{(r)}$ being absolutely continuous,

$$
\vert \hat{f}(k) \vert = \mathcal{O} \left( \frac{1}{|k|^{r+1}} \right)
$$
\end{theorem}

\textbf{Proof}. Note that 

$$
\begin{aligned}
 &= \\[8pt]
\end{aligned}
$$





\subsection*{Examples}

Let $f(t) := e^{-at} \mathbf{1}_{t>0}$ for $a >0$. Then

$$
\begin{aligned}
\hat{f}(\omega) &= \frac{1}{\sqrt{2\pi}} \int_0^\infty e^{-ax} e^{-i \omega t} dt \\[8pt]
\end{aligned}
$$

since $e^{-at - i\omega t} = e^{-at} \left( cos(-\omega t) + i sin( - \omega t) \right) $ 



\subsection*{DFT and FFT}


We have the following  of DFT. 



\begin{theorem}
\textbf{Circular shift property}. For some $0 \le m \le N-1$, define the shift by $m$ as follows:

$$
y_m[n] = \begin{cases} 
x[n-m+N] & 0 \le n < m \\
x[n-m] & m \le n \le N-1
\end{cases}
$$

Then we have the DFT of the circular shift $y_m[n]$ of $x[n]$ by $m$ is given by

$$
Y_m[k] = X[k] e^{-i \frac{2\pi km}{N}}
$$
\end{theorem}




\begin{theorem}
\textbf{Theorem}. Assume that $N=N_1N_2$ and the data $x[n], n=0,\cdots,N-1$, are of $N_1$-periodic. Suppose that $Y[l], l=0,\cdots,N_1-1$ are the DFTs of $N_1$ data $x[n], n=0,\cdots,N_1-1$. Then the DFT of $x[n]$ is given by

$$
X[k] = \begin{cases} 
N_2 Y[l] & k=1N_2, l=0,\cdots,N_1-1 \\
0 & o.w.
\end{cases}
$$
\end{theorem}


Assume that $N=N_1N_2$ and the data $x[n], n=0,1,\cdots,N-1$ are of $N_1$-periodic. We then have the following decomposition of DFT $X[k], k=0,\cdots,N$

$$
\begin{aligned}
X[k] &= \sum_{n=0}^{N-1} x[n] e^{-i\frac{2\pi kn}{N}} \\[8pt]
&= \sum_{n_1=0}^{N_1-1} \sum_{n_2=0}^{N_2-1} x[n_2 N_1 + n_1] e^{-i \frac{2\pi k (n_2 N_1 + n)1 }{N}} \\[8pt]
&= 
\end{aligned}
$$

where for each $n_1$,

$$
\sum_{n_2=0}^{N_2-1} x[n_2 N_1 + n_1] e^{-i \frac{2\pi kn_2}{N_2}}
$$

is a Fourier transform of $N_2$ data

$$
x[n_2 N_1 + n_1], n_2 = 0,\cdots, N_2 -1
$$

Since $x[n]$ is $N_1$-periodic, we have

$$
x[n_2 N_1 + n_1] = x[]
$$

thus,


$$
\begin{aligned}
X[k] &= N_2 \sum_{n_1=0}^{N_1 -1} x[n_1] e^{-i\frac{2\pi k n_1}{N}} \\[8pt]
&= N_2 \sum_{n_1=0}^{N_1-1} x[n_1] e^{-i \frac{2\pi k n_1}{N_1}} \\[8pt]
&= N_2 Y[l]
\end{aligned}
$$

where $Y[l]$ is the DFT of $N_1$ data $x[n], n=0,\cdots,N_1-1$. 



Note that from $\beta_j := \frac{1}{N} \sum_{k=0}^{N-1}$

$$
\begin{aligned}
\beta_j &= \frac{1}{N} \left( f_0 + f_1 e^{-ijx_1} + \cdots + f_{N-1} e^{-i(N-1)x_1} \right) \\[8pt]
&= 
\end{aligned}
$$

where


Start with $m=1$. Let $M:= 2^{m-1}$, $R:= 2^{n-1}$, for $m=1,2,\cdots$. From $2R(m) = 2^n$ phase polynomial $P_r^{(0)}(x)$, which are constant functions for $m=1, r=0,1,\cdots,2R-1$, find $R(m)$ phase polynomials $P_r^{(1)}, r=0,1,2,\cdots,R-1$ by

$$
P_r^{(1)} = \beta_{r,0}^{(1)} + \beta_{r,1}^{(1)} e^{ix}
$$

using the fact that 

$$
P_r^{(1)}
$$


From the $2R(m) = 2^{n-1}$ phase polynomials $P_r^{(1)} (x)$ for $r=0,1,2,\cdots,2R-1$, find $R(m) = 2^{n-1}$ phase 









As the following algorithm: 


\subsection*{Examples}


The Fourier series of $f(x):= sin^5(x)$ is as follows :





\end{document}