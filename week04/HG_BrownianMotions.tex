\documentclass[12pt]{article} 

\usepackage{geometry}
\geometry{a4paper} 

\usepackage{graphicx} 
\usepackage{enumitem}
\usepackage{booktabs}

\usepackage{float} 
\usepackage{wrapfig} 

\usepackage{amsmath}
\usepackage{amsfonts}
\usepackage{amssymb}
\usepackage{dsfont}

\usepackage{xcolor}
\usepackage{listings}
\usepackage{caption}
\DeclareCaptionFont{white}{\color{white}}
\DeclareCaptionFormat{listing}{%
  \parbox{\textwidth}{\colorbox{gray}{\parbox{\textwidth}{#1#2#3}}\vskip-2pt}}
\captionsetup[lstlisting]{format=listing,labelfont=white,textfont=white}
\lstset{frame=lrb,xleftmargin=\fboxsep,xrightmargin=-\fboxsep}

\linespread{1.2} 
\setlength{\parskip}{\baselineskip} % vertical spaces
\setlength\parindent{0pt} % remove all indentation from paragraphs


\usepackage{ntheorem}
\usepackage{mdframed}

\theoremstyle{nonumberbreak}
\theoremheaderfont{\bfseries}
\newmdtheoremenv[%
linecolor=gray,leftmargin=10,%
rightmargin=10,
backgroundcolor=gray!20,%
innertopmargin=0pt,%
ntheorem]{theorem}{}




\begin{document}

\title{\textbf{Brownian Motions}}
\author{Hyunwoo Gu}
\date{}

\maketitle


%----------------------------------------------------------------------------------------
%   Section 0
%----------------------------------------------------------------------------------------

\section{Joint Probabilities for Brownian Motion}

\begin{theorem}
\textbf{Theorem 2.1.} The conditional density of $X(t)$ for $t_1 < t < t_2$ given $X(t_1) = A$, $X(t_2) = B$ is a normal density with the mean

$$
A + \frac{B-A}{t_2 - t_1} (t - t_1)
$$

and the variance

$$
\frac{(t_2 - t) (t-t_1)}{ t_2 - t-1}
$$
\end{theorem}




\textbf{Proof}. Let $H$ be the interoccurrence distribution for $N(t)$. Then 


\subsection{Continuity of Paths and the Maximum Variables}

The physical origins of the Brownian motion process suggest that the possible realizations $X(t)$, (i.e. \textbf{sample path}) whose movements result from continuous collisions in the surronding medium are continuous functions. 








\begin{theorem}
\textbf{Theorem 3.1.} The probability that $X(t)$ has at least one zero in the interval $(t_0, t_1)$, given $X(0) = 0$, is

$$
\alpha = \frac{2}{\pi} \mathrm{arccos} \sqrt{t_0/t_1}
$$
\end{theorem}


\subsection{Variations and Extensions}

If $X(t)$ is a standard Brownian motion process, then the processes 

$$
\begin{aligned}
X_1(t) &= c X(t/c^2) \\[8pt]
X_2(t) &= \begin{cases}
t X(1/t) & t>0 \\
0 & t=0
\end{cases} \\[8pt]
X_3(t) &= X(t+h) - X(h) 
\end{aligned}
$$

for $c>0, h>0$.  




\subsection{Brownian Motion Absorbed at the Origin}






\end{document}



