\documentclass[12pt]{article} 

\usepackage{geometry}
\geometry{a4paper} 

\usepackage{graphicx} 
\usepackage{enumitem}
\usepackage{booktabs}

\usepackage{float} 
\usepackage{wrapfig} 

\usepackage{amsmath}
\usepackage{amsfonts}
\usepackage{amssymb}
\usepackage{dsfont}

\usepackage{xcolor}
\usepackage{listings}
\usepackage{caption}
\DeclareCaptionFont{white}{\color{white}}
\DeclareCaptionFormat{listing}{%
  \parbox{\textwidth}{\colorbox{gray}{\parbox{\textwidth}{#1#2#3}}\vskip-2pt}}
\captionsetup[lstlisting]{format=listing,labelfont=white,textfont=white}
\lstset{frame=lrb,xleftmargin=\fboxsep,xrightmargin=-\fboxsep}

\linespread{1.2} 
\setlength{\parskip}{\baselineskip} % vertical spaces
\setlength\parindent{0pt} % remove all indentation from paragraphs


\usepackage{ntheorem}
\usepackage{mdframed}

\theoremstyle{nonumberbreak}
\theoremheaderfont{\bfseries}
\newmdtheoremenv[%
linecolor=gray,leftmargin=10,%
rightmargin=10,
backgroundcolor=gray!20,%
innertopmargin=0pt,%
ntheorem]{theorem}{}




\begin{document}

\title{\textbf{Limit theorems for Markov Chains}}
\author{Hyunwoo Gu}
\date{}

\maketitle


%----------------------------------------------------------------------------------------
%   Section 1
%----------------------------------------------------------------------------------------
\begin{theorem}
\textbf{Theorem 1.2(The basic limit theorem of Markov chains)}. 

(a) Consider a \textbf{irreducible}, \textbf{ergodic}(recurrent and aperiodic) Markov chain. Let $P_{ii}^n$ the probability of entering state $i$ at the $n$th transition, $n=0,1,2,\cdots$, given that $X(O) = i$ (the initial state is $i$) . By our earlier convention $P_{ii}^0 = 1$. Let $f_{ii}^n$ the probability of first returning to state $i$ at the $n$th transition, $n=0,1,2,\cdots$ where $f_{ii}^0 = 0$. Thus 

$$
P^n_{ii} - \sum_{k=0}^n f_{ii}^{n-k} P_{ii}^k = \begin{cases}
1 & n=0 \\ 
0 & n>0
\end{cases}
$$

Then,

$$
\mathrm{lim}_{n\to\infty} P_{ii}^n = \frac{1}{\sum_{n=0}^\infty n f_{ii}^n}
$$

(b) Under the same conditions as in (a), 

$$
\mathrm{lim}_{n\to\infty} P^n_{ji} = \mathrm{lim}_{n\to\infty} P_{ii}^n.
$$
\end{theorem}


\textbf{Proof}. 


Under these circumstances we prove that $\mathrm{lim}_{n\to\infty} \gamma_n = c$. In fact,

$$
\begin{aligned}
y_n - c &= \sum_{k=0}^n a_{n-k} x_k - c\sum_{m=0}^\infty a_m \\[8pt]
&= \sum_{k=0}^n a_{n-k} (x_k -c) - c \sum_{m=n+1}^\infty a_m
\end{aligned}
$$

For $\epsilon >0$ prescribed we determine $K(\epsilon)$ so that $\vert x_k - c \vert < \epsilon / 3$ for all $k \ge K(\epsilon)$. 

$$
y_n - c = \sum_{k=0}^{K(\epsilon)} a_{n-k} (x_k -c) + \sum_{k=K(\epsilon)+1}^n a_{n-k} (x_k -c) - c \sum_{m=n+1}^\infty a_m
$$

and so 

$$
\vert y_n - c \vert \le \sum_{k=0}^{K(\epsilon)} a_{n-k} (x_k -c) + \sum_{k=K(\epsilon)+1}^n a_{n-k} (x_k -c) - c \sum_{m=n+1}^\infty a_m
$$

where 

$$
M = \mathrm{max}_{k \ge 0} \vert x_k - c \vert
$$


We choose $N(\epsilon)$ so that $|c| \sum_{m=n+1}^\infty a_m < \epsilon /3$ and 

$$
\sum_{k=0}^{K(t)} a_{n-k} \equiv \sum_{m=n-K(t)}^n a_m < \frac{\epsilon}{3M} \ \ n \ge N(\epsilon)
$$

Then 


\textbf{Remark 1.3.} Let $C$ be a recurrent class. Then $P_{ij}^n = 0$, for $i \in C, j \notin C$, and every $n$. Hence, once in $C$ it i snot possible to leave $C$. 


\begin{theorem}
\textbf{Theorem 1.3.} In a \textbf{positive recurrent aperiodic} class with states $j=0,1,2,\cdots$, 

$$
\mathrm{lim}_{n\to \infty} P^n_{jj} = \pi_j = \sum_{i=0}^\infty \pi_i P_{ij}, \ \ \sum_{i=0}^\infty \pi_i =1
$$

and the $\pi$'s are uniquely determined by the set of equations
\end{theorem}



\subsection{Absorption Probabilities}

\textbf{Remark 3.1.} If there are only a finite number of states, M, then there
are no null states and not all states can be transient . In fact, since
Lf 01 Pij == 1 for all n, it cannot happen that limn-+ oo Pij = 0 for all j.
The same argument restricted to recurrent classes shows that there are
no null states . Let C, C 1 , C2 , • . • denote recurrent classes . We define
ni(C) as the probability that the process will be ultimately absorbed into
the recurrent class C if the initial state is the transient state i. (Recall
that once the process enters a recurrent class, it never leaves it.)




\begin{theorem}
\textbf{Theorem 1.3.} In a \textbf{positive recurrent aperiodic} class with states $j=0,1,2,\cdots$, 

$$
\mathrm{lim}_{n\to \infty} P^n_{jj} = \pi_j = \sum_{i=0}^\infty \pi_i P_{ij}, \ \ \sum_{i=0}^\infty \pi_i =1
$$

and the $\pi$'s are uniquely determined by the set of equations
\end{theorem}



\textbf{Proof}. Clearly $\pi_i^n (C) = \sum_{k \in C} \pi_{ik}^n (C)$ where $\pi_{ik}^n(C)$ represents the probability starting from state $i$ of being absorbed at the $n$th transition into class $C$ at state $k$. We have

$$
\pi_i(C) = \sum_{\nu = 1 }
$$


Therefore $\forall \epsilon >0$, $\exists$ a finite number of states $C' \subset C$ and an integer $N(\epsilon) = N$ such that 

$$
\vert \pi_i(C) - \sum_{\nu=1}^n \sum_{k \in C'} \pi_{ik}^\nu (C) \vert < \epsilon
$$

i.e.



Combining these relations, we have 


\textbf{Example (The gambler's ruin on $n+1$ states)}. 




$$
P = \begin{bmatrix}
1 & 1 & 0 & 0 & \cdots & 0 \\
0 & 0 & 1 & 0 & \cdots & 0 \\
\vdots &  &  &  &  & \vdots \\
0 & 0 & 0 & 0 & \cdots & 1 \\
1 & 0 & 0 & 0 & \cdots & 0 \\
\end{bmatrix}
$$



\subsection{Criteria for Recurrence}

Let us prove theorems which will be useful in determining whether a
given Markov chain is recurrent or transient and then we apply them to
several examples

\begin{theorem}
\textbf{Theorem 4.1.} Let $\mathcal{B}$ be an irreducible Markov chain whose state space is labeled by the nonnegative integers. Then a necessary and sufficient condition that $\mathcal{B}$ be transient (i.e. each state is a transient state) is taht the system of equations

$$
\sum_{j=0}^\infty P_{ij} \gamma_j = \gamma_i, \ \ i\neq0
$$

have a bounded nonconstant solution.

\end{theorem}

\textbf{Proof}. For the sufficiency, let the transition matrix for $\mathcal{B}$ be

$$
P = \Vert P_{ij} \Vert = \begin{bmatrix}
\end{bmatrix}
$$

where the zero state has been converted into an absorbing barrier while the transition probabilities governing the motion among the other states are unchanged. Let us denote 




If the chain is recurrent, then 

$$
\mathrm{lim}_{n\to\infty} \tilde{P}_{i0}^n = 1
$$

and 


where



\begin{theorem}
\textbf{Theorem 4.2.} In an irreducible Markov chain a sufficient condition for recurrence is that there exists a sequence $\{ \gamma_i \}$ such that

$$
\sum_{j=0}^\infty P_{ij} \gamma_j \le \gamma_i \ \ i \neq 0, \gamma_i \to \infty
$$

\end{theorem}

\textbf{Proof}. Using the same notation as in the previous theorem, we have

$$
\sum_{j=0}^\infty \hat{P}_{ij} \gamma_j \le \gamma_i, \ \ \forall i
$$

Since 



Given $\epsilon$, we choose $M(\epsilon)$ such that $1/\gamma_i \le \epsilon$ for $i \ge M(\epsilon)$. Now

$$
\sum_{j=0}^{M-1} \tilde{P}_{ij}^m \gamma_j + \sum_{j=M}^\infty \tilde{P}_{ij}^m \gamma_j \le \gamma_i
$$

and so

$$
\sum_{j=0}^{M-1} \tilde{P}_{ij}^m \gamma_j + \mathrm{min}_{r \ge M} \{ \gamma_r \} \sum_{j=M}^\infty \tilde{P}_{ij}^m \le \gamma_i
$$


Since 

$$
\sum_{j=0}^\infty \tilde{P}_{ij}^m = 1
$$

we have

$$
\sum_{j=0}^{M-1} \tilde{P}_{ij}^m \gamma_j + \mathrm{min}_{r \ge M} \{ \gamma_r \} \left( 1 - \sum_{j=0}^{M-1} \tilde{P}_{ij}^m \right) \le \gamma_i
$$


As observed in the proof of the preceding theorem, 

$$
\mathrm{lim}_{n\to\infty} \tilde{P}^n_{ij} = 0, \ \ j\neq 0
$$

Thus passing to the limit as $m \to \infty $, we obtain for each $i$,

$$
\tilde{\pi}_i (C_0)\gamma_0 + \mathrm{min}_{r \ge M} \{ \gamma_r \} (1 - \tilde{\pi}_i (C_0)) \le \gamma_i
$$


\subsection{A Queueing Example}


If $\sum k a_k \le 1$, the process is recurrent.

In order to ascertain whether the process $\mathcal{B}$ is null recurrent or positive recurrent we first deal with the following auxiliary problem of some independent interest.

We set 

$$
G(s) = \frac{b_{-1}}{s} + b_0 + b_1 s + b_2 s^2 + \cdots
$$


Our is to determine $U(s)$ in terms of $G(s)$. To this end we write the usual renewal relations 

$$
\gamma_1 = b_{-1}, \ \ \gamma_k = \sum_{j=0}^\infty b_j \gamma_{j-1}^{(j+1)}, \ \ k \ge 2
$$

\end{document}