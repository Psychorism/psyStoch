\documentclass[12pt]{article} 

\usepackage{geometry}
\geometry{a4paper} 

\usepackage{graphicx} 
\usepackage{enumitem}
\usepackage{booktabs}

\usepackage{float} 
\usepackage{wrapfig} 

\usepackage{amsmath}
\usepackage{amsfonts}
\usepackage{amssymb}
\usepackage{dsfont}

\usepackage{xcolor}
\usepackage{listings}
\usepackage{caption}
\DeclareCaptionFont{white}{\color{white}}
\DeclareCaptionFormat{listing}{%
  \parbox{\textwidth}{\colorbox{gray}{\parbox{\textwidth}{#1#2#3}}\vskip-2pt}}
\captionsetup[lstlisting]{format=listing,labelfont=white,textfont=white}
\lstset{frame=lrb,xleftmargin=\fboxsep,xrightmargin=-\fboxsep}

\linespread{1.2} 
\setlength{\parskip}{\baselineskip} % vertical spaces
\setlength\parindent{0pt} % remove all indentation from paragraphs


\usepackage{ntheorem}
\usepackage{mdframed}

\theoremstyle{nonumberbreak}
\theoremheaderfont{\bfseries}
\newmdtheoremenv[%
linecolor=gray,leftmargin=10,%
rightmargin=10,
backgroundcolor=gray!20,%
innertopmargin=0pt,%
ntheorem]{theorem}{}




\begin{document}

\title{\textbf{The Queueing Theory}}
\author{Hyunwoo Gu}
\date{}

\maketitle


%----------------------------------------------------------------------------------------
%   Section 1
%----------------------------------------------------------------------------------------
\section{Definitions}

The \textbf{first} character describes the \textbf{arrival process} to the queue.

\begin{itemize}
	\item \textbf{M}emorylessness: Poisson arrival process\item \textbf{D}eterministic: interarrival interval is nonrandom
	\item \textbf{G}eneral: general interarrival distribution
\end{itemize}

The \textbf{second} character describes the \textbf{service process}, with $M$: exponentially distributed service times. 

The \textbf{third} character describes the \textbf{number of servers}.  

Sometimes a fourth character is added which gives the \textbf{number of customers} that can be saved in the queue plus service facility. 

It is often assumed that the service times are IID, independent of the arrival epochs, and independent of which server is used. 


\textbf{M/M/1 queue}(Gallager). A queueing system with a Poisson arrival system with rate $\lambda$ and a single server that serves arriving customers in order with a service time distribution $F(y) = 1 - exp(-\mu y)$. 

The service times are IID and independent of the interarrival intervals. During any period when the server is busy, customers leave the system according to a Poisson process



\textbf{M/G/$\infty$ queue}(Gallager). A queueing system with Poisson arrivals, a general service distribution, and an infinite number of servers. From the infinite number of servers, no arriving customers are ever queued. The service time $Y_i$ of customer $i$ is IID over $i$ with CDF $G(y)$. 





\end{document}